\documentclass{article}

\usepackage{Style_Thesis-Report}

\title{Probability}
\author{Ignacio Sada S\'{o}lomon}
\date{Winter 2020}  %Place'%' at front to activate today's date


\begin{document}
\clearpage\maketitle
\thispagestyle{empty}
\vspace{2cm}

\begin{abstract}
	Sample space, events, conditional probability, independence of events, Bayes' Theorem. Basic combinatorial probability, random variables, discrete and continuous univariate and multivariate distributions. Moment generating functions Independence of random variables. Chebyshev’s inequality,central limit theorem, weak law of large numbers (if time allows).
\end{abstract}

\newpage

\tableofcontents
\newpage
\setcounter{page}{1}
\cfoot{\thepage}

\section{Fundamentals}
\subsection{Set Theory}

	\begin{defn}
		A \textbf{set} is a collection of objects, called elements.
	\end{defn}
	\begin{exmp}
		The set of natural numbers $\N = \{ 1, 2, 3, \dots\}$, which is both infinite and countable.
	\end{exmp}
	\begin{exmp}
		The set of all McGill students.
	\end{exmp}
	\begin{defn}
		Let $S$ be a set. A \textbf{subset} is another set $A$ such that every element of $B$ is in $A$ as well:
		$$ A = \{ x: x \in S\} \quad \text{ or } \quad x \in A \implies x \in S$$
	\end{defn}
	\begin{rem}
		The empty set is denoted $\varnothing = \{\}$
	\end{rem}
	\begin{defn}
		An \textbf{intersection} between the sets A and B is defined as
		$$ A \cap B = x \in A \quad \text{and}\quad x \in B$$
	\end{defn}
	\begin{defn}
		A \textbf{union} between the sets A and B is defined as
		$$ A \cup B = x \in A \quad \text{or}\quad x \in B$$
	\end{defn}
	\begin{defn}
		Let $X$ be a universal set with a subset $A$. Then the complement of $A$ with respect to $X$ is
		$$ A^c = X \setminus A = \{ x: x \in X \quad \text{and} \quad x \notin A \}$$
	\end{defn}
	\begin{exmp}
		$\mathcal{P} (B) = \{ A: A \subseteq B \}$ (all subsets of $B$)
		
		Let $B = \{1,2,3\}$, then
		$$ \mathcal{P}(B) = \bigg\{ \underbrace{ \{\}, \{1,2,3\}}_{\text{trivial subsets}}, \underbrace{\overbrace{\{1\}, \{2\}, \{3\},}^{\text{singletons}} \overbrace{\{1,2\}, \{1,3\}, \{2,3\}}^{\text{doubles}}}_{\text{proper subsets}} \bigg\}$$
	\end{exmp}
	\begin{defn}
		Two sets $A$ and $B$ are denoted as \textbf{disjoint} whenever they share no elements, i.e. they have nothing in common:
		$$ A \cap B = \varnothing$$
	\end{defn}
\end{document}      
