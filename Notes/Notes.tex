\documentclass{article}

\usepackage{Style_Thesis-Report}

\title{Probability}
\author{Ignacio Sada S\'{o}lomon}
\date{Winter 2020}  %Place'%' at front to activate today's date


\begin{document}
\clearpage\maketitle
\thispagestyle{empty}
\vspace{2cm}

\begin{abstract}
	Sample space, events, conditional probability, independence of events, Bayes' Theorem. Basic combinatorial probability, random variables, discrete and continuous univariate and multivariate distributions. Moment generating functions Independence of random variables. Chebyshev’s inequality,central limit theorem, weak law of large numbers (if time allows).
\end{abstract}

\newpage

\tableofcontents
\newpage
\setcounter{page}{1}
\cfoot{\thepage}

\section{Fundamentals}
\subsection{Set Theory}

	\begin{defn}
		A \textbf{set} is a collection of objects, called elements.
	\end{defn}
	\begin{exmp}
		The set of natural numbers $\N = \{ 1, 2, 3, \dots\}$, which is both infinite and countable.
	\end{exmp}
	\begin{exmp}
		The set of all McGill students.
	\end{exmp}
	\begin{defn}
		Let $X$ be a set. A \textbf{subset} is another set $A$ such that every element of $A$ is in $X$ as well:
		$$ A = \{ x: x \in X\} \quad \text{ or } \quad x \in A \implies x \in X$$
	\end{defn}
	\begin{rem}
		The empty set is denoted $\varnothing = \{\}$
	\end{rem}
	\begin{defn}
		An \textbf{intersection} between the sets A and B is defined as
		$$ A \cap B = x \in A \quad \text{and}\quad x \in B$$
	\end{defn}
	\begin{defn}
		A \textbf{union} between the sets A and B is defined as
		$$ A \cup B = x \in A \quad \text{or}\quad x \in B$$
	\end{defn}
	\begin{defn}
		Let $X$ be a universal set with a subset $A$. Then the complement of $A$ with respect to $X$ is
		$$ A^c = X \setminus A = \{ x: x \in X \quad \text{and} \quad x \notin A \}$$
	\end{defn}
	\begin{exmp}
		$\mathcal{P} (B) = \{ A: A \subseteq B \}$ (all subsets of $B$)
		
		Let $B = \{1,2,3\}$, then
		$$ \mathcal{P}(B) = \bigg\{ \underbrace{ \{\}, \{1,2,3\}}_{\text{trivial subsets}}, \underbrace{\overbrace{\{1\}, \{2\}, \{3\},}^{\text{singletons}} \overbrace{\{1,2\}, \{1,3\}, \{2,3\}}^{\text{doubles}}}_{\text{proper subsets}} \bigg\}$$
	\end{exmp}
	\begin{defn}
		Two sets $A$ and $B$ are denoted as \textbf{disjoint} whenever they share no elements, i.e. they have nothing in common:
		$$ A \cap B = \varnothing$$ 
	\end{defn}
	\begin{exmp}
		Let $\Omega = \N = \{ 0,1,2,3,4,\dots\}$ and $A \subseteq N = \{ n \in \N: n \text{ is even}\}$. Then,
		$$ A^c = \{ n \in \N : n \text{ is odd}\}$$
	\end{exmp}
\pagebreak
\subsubsection{Properties of Sets}
	\begin{multicols}{2}
		\begin{itemize}
			\item $A \cap \varnothing = \varnothing$
			\item $A \cup \varnothing = A$
			\item $A \cup B = B \cup A$
			\item $A \cap B = B \cap A$
			\item $(A\cap B)\cap C = A \cap (B \cap C)$
			\item $(A \cup B) \cup C = A \cup (B \cup C)$
			\item $(A \cap B)^c = A^c \cup B^c$
			\item $(A \cup B)^c = A^c \cap B^c$
		\end{itemize}
	\end{multicols}
	\begin{exe}
		Prove that
		\begin{enumerate}[$\quad \quad$a)]
			\item $A \cup (B \cap C) = (A \cup B) \cap (A \cup C)$
			\item $A \cap (B \cup C) = (A \cap B) \cup (A \cap C)$
		\end{enumerate}
	\end{exe}
	\begin{note}
		Consider $A \setminus B = \{ x\in A: x \notin B \}$. Then we define the \textbf{exclusive or} as
		$$ A \setminus B \cup B \setminus A = A \triangle B$$
	\end{note}
\pagebreak
\section{Probability}
	Probability is the branch of applied mathematics that deals with random events, which are non-deterministic in nature, unlike calculus for instance, which deals with deterministic events.
	\begin{exmp}
		Toss a coin. We won't know the outcome of the toss, but we do know that the outcome will be either \emph{heads} or \emph{tails}.
	\end{exmp} 
	\begin{exmp}
		Roll a die. We now have several outcomes, depending on the number of faces on the die.
	\end{exmp}
	\begin{defn}
		Given a random experiment, the set of all possible outcomes is denoted $\Omega$, and is called the \textbf{sample space}.
	\end{defn}
	\begin{exmp} $\quad$ \\
		\vspace{-0.5cm}
		\begin{itemize}
			\item The sample space for a coin toss is $\Omega = \{ H, T \}$
			\item The sample space for a rolled 6-face die is $\Omega = \{1,2,3,4,5,6\}$
			\item The sample space for a coin toss where we require heads once is $\Omega = \{ H, TH, TTH, \dots \}$
		\end{itemize}
	\end{exmp}
	\begin{defn}
		Let $\Omega$ denote the set of all possible outcomes of a random experiment.
		\begin{enumerate}[$\quad\,\,${2.2}.1.]
			\item An \textbf{elementary event} is a subset of $\Omega$ with a singular element (i.e. one single outcome) and is also an element of $\mathcal{P}(\Omega)$.
			\item A \textbf{compound event} is any subset of $\Omega$ (including the elementary events).
			\item An \textbf{event} is any element of $\Omega$.
			\item $\varnothing$ denotes \textbf{impossible events}.
			\item A \textbf{complimentary event} given an event $A$ is denoted $A^c$.
			\item Given 2 events $A$ and $B$, where $A \cap B = \varnothing$, we denote them as \textbf{disjoint events}.
		\end{enumerate}
	\end{defn}
	\begin{defn}
		Let $\Omega$ be the sample space attached to a random experiment. A \textbf{probability} is a function $P$ such that
		$$ P: \mathcal{P}(\Omega) \to [0,1]$$
	\end{defn}
	Note that $P(\Omega)=1$. Given a sequence $A_1, \dots, A_n, \dots$, of pairwise disjoint events ($A_i \cap A_j = \varnothing \quad \forall i \neq j$), we have that 
	$$ P \left(  \bigcup_{i =1} A_i\right) = \sum_{i =1} P (A_i)$$
	\begin{rem}
		In general, if $\Omega$ is countable, then it is enough to define $P$ on the elementary event, as every other event can be written as a union.
	\end{rem}
	\begin{exmp}
		Toss a coin such that $\Omega = \{ H, T \}$. A probability on $\Omega$ is completely given by $p \in [0,1]$, and the assignment of $P(\{H\}) = p$, and $P(\{T\}) = 1-p$.
	\end{exmp}
	\begin{rem}
		If $p=1/2$ in the previous example, then $P(\{H\}) = P(\{T\}) = 1/2$, implying that \emph{the coin is fair/balanced}.
	\end{rem}
	\begin{exmp}
		Throw a die such that $\Omega = \{ \omega_1, \omega_2, \omega_3, \omega_4, \omega_5, \omega_6 \}$. A probability on $\Omega$ is given by 6 non-negative integers $p_i$ where $i=1,2,3,4,5,6$, such that
		$$ p_i = P(\{w_i\}) \geq 0$$
		And so
		$$ \sum_{i=1}^6 p_i = 1$$
		Therefore, if we consider the die to be fair, we have $p_i = 1/6 \quad \forall i \in \{ 1,2,3,4,5,6 \}$. \\ Let $A = \{ \omega_1, \omega_3\}$ Then
		\begin{align*}
		 A &= \{ \omega_1\} \cup \{ \omega_3 \} \implies P(A) \\ 
		 &= P(\{ \omega_1 \}) + P(\{ \omega_3 \}) \\
		 &= 1/6 + 1/6 \\
		 &= 1/3
		\end{align*} 
	\end{exmp}
	\begin{exe}
		Consider a die such that the probability $P(\{\omega_i\})$ is proportional to $k$ such that
		$$ P \big( \{\omega_i\} \big) = ck$$
		Therefore:
		\begin{align*}
			1 &= \sum_{k=1}^6 P(\{\omega_k\}) \\
			&= \sum_{k=1}^6 ck \\
			&= c\sum_{k=1}^6 k \\
			&= c\frac{n(n+1)}{2}\bigg|_{n=6} \\
			&= c \frac{6 \times 7}{2} \\
			&= 21c\\
		\intertext{ }
			\therefore c &= \frac{1}{21}
		\end{align*}
		
		What would be the probability of an even number?\\
		Let $A = \{ \omega_2, \omega_4, \omega_6 \}$, and so
		\begin{align*}
			P(A) &= P(\{\omega_2\}) + P(\{\omega_4\}) + P(\{\omega_6\})  \\
			&= \frac{2+4+6}{21} \\
			&= \frac{12}{21}
		\end{align*}
		
		What would be the probability of an odd number?\\
		Simple:
		\begin{align*}
			P(B) &= 1- P(A) \\
			&= \frac{9}{12}
		\end{align*}
	\end{exe}
\end{document}      
