\documentclass{article}

\usepackage{Style_Thesis-Report}

\title{Math 323
	- Assignment 1}
\author{Sada S\'{o}lomon, Ignacio - Id 260708051}
\date{Winter 2020}  %Place'%' at front to activate today's date


\begin{document}
\clearpage\maketitle
\thispagestyle{empty}
\vspace{2cm}

\newpage
\setcounter{page}{1}
\cfoot{\thepage}

\pagebreak

\section*{Problem 1}
	\emph{Suppose a family contains two children of different ages, and we are interested in the gender of these children. Let $F$ denote a female child, and $M$ denote male child, such that the ordered pair $FM$ denotes an older female child and a younger male child without loss of generality. In the set $S$, we thus have: $$ S = \{ FF, \, FM, \, MF, \, MM \}$$ Let $A$ denote the subset of possibilities containing no males, $B$ denote the subset containing two males, and $C$ the subset containing at least one male. List the elements of $A, B, C, A \cap B, A \cup B, A \cap C, A \cup C, B \cap C, B \cup C,$ and $C \cap \bar{B}$. }
	
	\begin{sol}
		\vspace{-1cm}
		\begin{align*}
			A &= \{ FF \} \\
			B &= \{ MM \} \\
			\bar{B} &= \{ FF, \, FM, \, MF\} \\
			C &= \{ FM, \, MF, \, MM \} \\
			A\cap B &= \{  \} = \varnothing \\
			A \cup B &= \{ FF, \, MM\} \\
			A \cap C &= \{ \} = \varnothing \\
			A \cup C &= \{ FF, \, FM, \, MF, \, MM\} = S \\
			B \cap C &= \{ MM \} = B \\
			B \cup C &= \{ FM, \, MF, \, MM \} = C \\
			C \cap \bar{B} &= \{ FM, \, MF\}
		\end{align*}
	\end{sol}
\pagebreak
\section*{Problem 2}
	\emph{Suppose two dice are tossed, and the numbers on he upper faces are observed. Let $S$ denote the set of all possible pairs that can be observed. [These pairs can be listed, for example, by letting $(2,3)$ denote that a 2 was observed on the first die, and a 3 on the second.]
	\begin{enumerate}[$\quad$a)]
		\item Define the following subsets of $S$: 
		\vspace{-0.5cm}
			\begin{align*}
				A: &\text{The number on the second die is even.}\\
				B: &\text{The sum of the two numbers is even. }\\
				C: &\text{At least one number in the pair is odd. }
			\end{align*}
		\item List the points in $A, \bar{C}, A \cap B, A \cap \bar{B}, \bar{A} \cup {B},$ and $\bar{A} \cap C$
	\end{enumerate}
	}
	\begin{sol}
		\vspace{-0.6cm}
		\begin{enumerate}[$\quad$a)]
			\item We begin by defining the a particular subset of the natural numbers $\N_k$. Let $k$ be a positive integer such that: $$ \N_k = \{1, 2, \dots, k\} $$
			The particular subset $N_k$ has the following multiplicative property for $\alpha \in \N$:
			$$ \N_{\alpha (k) } = \{ \alpha (1), \alpha(2), \dots, \alpha (k) \}$$
			We can thus define the set $S = \{ (a, b): a, b \in \N_6\}$ where $a$ denotes the result of the first die, and $b$ denotes the result of the second die. Now, we can use the same notation to define the required subsets:
			\begin{align*}
				A &= \{ (a, b) : b \in \N_{2(3)} \}\\
				B &= \{ (a, b) : a+b \in \N_{2(6)} \} \\
				C &= \{ (a, b) : a \notin \N_{2(3)}\quad \nabla \quad  b \notin \N_{2(3)}   \}
			\end{align*}
			Note the use of $\nabla$ (XOR), which is the exclusive $\lor$ (OR).
		\pagebreak
			\item 
			\begin{align*}
				A &= \{ (1, 2), (1, 4), (1, 6), (2, 2), (2, 4), (2, 6), (3, 2), (3, 4), (3, 6), (4, 2), (4, 4), (4, 6), \\
				&\quad\quad\quad\quad\quad\quad (5, 2), (5, 4), (5, 6), (6, 2), (6, 4), (6, 6) \}\\
				\bar{C} &= \{ (2,2), (2,4), (2,6), (4,2), (4,4), (4, 6), (6, 2), (6, 4), (6,6) \} \\
				A \cap B &=\{ (1, 2), (1, 4), (1, 6), (2, 2), (2, 4), (2, 6), (3, 2), (3, 4), (3, 6), (4, 2), (4, 4), (4, 6), \\
				&\quad\quad\quad\quad\quad\quad (5, 2), (5, 4), (5, 6), (6, 2), (6, 4), (6, 6) \} \cap \{ (1,1), (1,3), (1, 5), (2,2), \\
				&\quad\quad\quad\quad\quad\quad (2,4), (2,6), (3, 1), (3, 3), (3, 5), (4, 2), (4, 4), (4, 6), (5, 1), (5, 3), (5,5), \\
				&\quad\quad\quad\quad\quad\quad (6, 2), (6, 4), (6, 6) \}\\
				&= \{ (2,2), (2,4), (2,6), (4,2), (4,4), (4,6), (6,2), (6,4), (6,6) \} = \bar{C} \\
				A \cap \bar{B} &= \{ (1, 2), (1, 4), (1, 6), (2, 2), (2, 4), (2, 6), (3, 2), (3, 4), (3, 6), (4, 2), (4, 4), (4, 6), \\
				&\quad\quad\quad\quad\quad\quad (5, 2), (5, 4), (5, 6), (6, 2), (6, 4), (6, 6) \} \cap \{ (1, 2), (1, 4), (1, 6), (2, 1),\\
				&\quad\quad\quad\quad\quad\quad (2, 3), (2, 5), (3, 2), (3, 4), (3, 6), (4, 1), (4, 3), (4, 5), (5, 2), (5, 4), (5, 6),\\
				&\quad\quad\quad\quad\quad\quad (6, 1), (6, 3), (6, 5)\}\\
				&= \{ (1,2), (1,4), (1, 6), (3, 2), (3, 4), (3, 6), (5, 2), (5, 4), (5, 6) \}  \\
				\bar{A} \cup B &= \{ (1, 1), (1, 3), (1, 5), (2, 1), (2, 3), (2, 5), (3, 1), (3, 3), (3, 5), (4, 1), (4, 3), (4, 5), \\
				&\quad\quad\quad\quad\quad\quad (5, 1), (5, 3), (5, 5), (6, 1), (6, 3), (6, 5) \} \cup \{ (1,1), (1,3), (1, 5), (2,2), \\
				&\quad\quad\quad\quad\quad\quad (2,4), (2,6), (3, 1), (3, 3), (3, 5), (4, 2), (4, 4), (4, 6), (5, 1), (5, 3), (5,5), \\
				&\quad\quad\quad\quad\quad\quad (6, 2), (6, 4), (6, 6) \}\\
				&= \{ (1, 1), (1, 3), (1,5), (2, 1), (2,2), (2,3), (2,4), (2,5), (2,6), (3,1), (3,3), (3,5), \\
				&\quad\quad\quad\quad\quad\quad (4,1), (4,2), (4,3), (4, 4), (4, 5), (4,6), (5,1), (5,3), (5,5), (6,1), (6,2), \\
				&\quad\quad\quad\quad\quad\quad(6,3), (6,4), (6,5), (6,6) \}\\
				\bar{A} \cap C &=  \{ (1, 1), (1, 3), (1, 5), (2, 1), (2, 3), (2, 5), (3, 1), (3, 3), (3, 5), (4, 1), (4, 3), (4, 5), \\
				&\quad\quad\quad\quad\quad\quad (5, 1), (5, 3), (5, 5), (6, 1), (6, 3), (6, 5) \} \cap \{ (1, 1), (1,2), (1, 3), (1, 4), \\
				&\quad\quad\quad\quad\quad\quad (1, 5), (1,6), (2, 1), (2, 3), (2, 5), (3, 1), (3,2), (3, 3), (3, 4), (3, 5), (3, 6), \\
				&\quad\quad\quad\quad\quad\quad (4, 1), (4, 3), (4, 5), (5,1), (5,2), (5, 3),(5, 4), (5, 5), (5, 6), (6, 1), (6, 3),\\
				&\quad\quad\quad\quad\quad\quad (6, 5) \} \\
				&= \{ (1,1), (1,3), (1,5), (2, 1), (2, 3), (2, 5), (3, 1), (3, 3), (3, 5), (4, 1), (4, 3), (4,5), \\
				&\quad\quad\quad\quad\quad\quad (5, 1), (5, 3), (5, 5), (6, 1), (6, 3), (6, 5) \} = \bar{A}
			\end{align*}
		\end{enumerate}
	\end{sol}
\pagebreak
\section*{Problem 3}
	\emph{A survey classified a large number of adults according to whether they were diagnosed as needing eyeglasses to correct their reading vision, and whether they use eyeglasses when reading. The proportions falling into the four resulting categories are given in the following table:}
	\begin{table}[h]
		\begin{tabular}{c|c|c|}
			\cline{2-3}
			& \multicolumn{2}{c|}{\begin{tabular}[c]{@{}c@{}}Uses Eyeglasses \\ for Reading?\end{tabular}} \\ \hline
			\multicolumn{1}{|c|}{Needs glasses?} & Yes  & No   \\ \hline
			\multicolumn{1}{|c|}{Yes}            & \, 0.44 \,\, & 0.14 \\ \hline
			\multicolumn{1}{|c|}{No}             & \, 0.02 \,\, & 0.40 \\ \hline
		\end{tabular}
	\end{table}

	\emph{If a single adult is selected from the large group, find the probabilities of the events defined below. \\ The adult
	\begin{enumerate}[$\quad$a)]
		\item needs glasses.
		\item needs glasses but does not use them.
		\item uses glasses whether the glasses are needed or not.
	\end{enumerate}	
	}
	\begin{sol}
		\vspace{-0.7cm}
		\begin{enumerate}[$\quad$a)]
			\item An adult from the large group that needs glasses has a probability of $0.44+0.14 = 0.60$. 
			\item An adult from the large group that needs glasses, but does not use them, has a probability of $0.14$.
			\item An adult from the large group that uses glasses regardless of whether they are needed or not has a probability $0.44+0.02 = 0.46$.
		\end{enumerate}
	\end{sol}
\pagebreak
\section*{Problem 4}
	\emph{An oil prospecting firm hits oil or gas on 10\% of its drillings. If the firm drills two wells, the four possible simple events and three of their associated probabilities are as given in the accompanying table: }
	\begin{table}[h]
		\begin{tabular}{|c|c|c|c|}
			\hline
			\begin{tabular}[c]{@{}c@{}}Simple\\ Event\end{tabular} &
			\begin{tabular}[c]{@{}c@{}}Outcome of\\ First Drilling\end{tabular} &
			\begin{tabular}[c]{@{}c@{}}Outcome of\\ Second Drilling\end{tabular} &
			Probability \\ \hline
			$E_1$ & Hit  & Hit  & 0.01 \\ \hline
			$E_2$ & Hit  & Miss & -    \\ \hline
			$E_3$ & Miss & Hit  & 0.09 \\ \hline
			$E_4$ & Miss & Miss & 0.81 \\ \hline
		\end{tabular}
	\end{table}

	\emph{Find the probability that the company will hit oil or gas
		\begin{enumerate}[$\quad$a)]
			\item On the first drilling and miss on the second.
			\item On at least one of the two drilling.
		\end{enumerate}
	} 
	\begin{sol}
		\begin{enumerate}[$\quad$a)]
			\item The company will hit oil or gas on the first drilling and miss on the second with a probability of $1-0.01-0.09-0.81=0.09$
			\item The company will hit oil or gas on at least one of the two drillings with a probability of $1 - 0.81 = 0.19$. Also note that this probability can be found through $0.01+0.09+0.09=0.19$.
		\end{enumerate}
	\end{sol}
\pagebreak
\section*{Problem 5}
	\emph{Extend Theorem 5, proved in class, to three events, A, B, and C, by finding an expression for $P(A \cup B \cup C)$ in terms of the probabilities of $A, B$, and $C$, of their pair-wise intersections, and the intersection of all three events. (Hint: Begin by considering $A \cup B$ as a single event). }
\pagebreak
\section*{Problem 6}
	\emph{Here is a subtle question. Criticize the reasoning of Example 2.3 p37 given by Wackerly, Mendenhall, and Scheaffer. They argue that just because the coin is balanced, each outcome (the result of three tosses) must have probability $1/8$. Note that a coin is balanced if $P(H) = P(T) = 1/2$ at each toss. Is is enough to criticize the reasoning by considering just a two-toss experiment, for which WMS would (incorrectly) argue that the probability of each pair of outcome is automatically $1/4$, from the fairness of the coin alone. This mistake is made in many introductory statistics books. (This is not complicated, but you will need to look at your notes carefully before submitting.)
	}
\pagebreak
\section*{Problem 7}
	\emph{We previously considered a situation where cars entering an intersection each could turn right, turn left, or go straight. An experiment consists of observing two vehicles moving through the intersection. 
	\begin{enumerate}[$\quad$a)]
		\item How many sample points are there in the sample space?
		\item Assuming that all sample points are equally likely, what is the probability that at least one vehicle turns left?
		\item Again assuming all sample points are equally likely, what is the probability that at most one vehicle turns?
	\end{enumerate}
	}	
	\begin{sol}
		\begin{enumerate}[$\quad$a)]
			\item There are 2 vehicles, and each car has 3 possible directions in which to turn. Thus, the total sample points can be calculated by:
			$$ (3 \text{ directions})^{(2 \text{ vehicles})} = 9 \text{ sample points}$$
			Let us list these 9 points through pairs $(A, B)$ where $A$ denotes the direction of the first car and $B$ the direction of the second. We have:
			\begin{multicols}{3}
				$(Left, \, Left)$ \\
				$(Left, \, Straight)$ \\
				$(Left, \, Right)$ \\
				$(Straight,\, Left)$ \\
				$(Straight,\, Straight)$ \\
				$(Straight,\, Right)$ \\
				$(Right,\, Left)$ \\
				$(Right,\, Straight)$ \\
				$(Right,\, Right)$
			\end{multicols}
			\item We can see above that out of all 9 possibilities, only 5 of them have at least one vehicle turning left. Therefore, the probability that at least one vehicle turns left is $5/9$. We can also see that the amount of possibilities where no vehicle turns left at all is $4/9$, and so $1 - 4/9 = 5/9$.
			
			\item We again have a total 9 possibilities, only that this time we consider the cases where both of the vehicles turn either left or right. Thus, we can see that there are only 4 cases where this occurs. As we are looking for the number of possibilities where at most one car turns, we calculate this again by subtracting the number of cases where both vehicles turn from all possibilities: $1- 4/9 = 5/9$. If we count the number of possibilities that include a straight direction (of which at most one car turns), we see that there are 5 possibilities, confirming the result.
		\end{enumerate}
		
	\end{sol}
\end{document}  

	