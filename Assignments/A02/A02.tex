\documentclass{article}

\usepackage{Style_Thesis-Report}

\title{Math 323
	- Assignment 2}
\author{Sada S\'{o}lomon, Ignacio - Id 260708051}
\date{Winter 2020}  %Place'%' at front to activate today's date


\begin{document}
\clearpage\maketitle
\thispagestyle{empty}
\vspace{2cm}

\newpage
\setcounter{page}{1}
\cfoot{\thepage}

\pagebreak

\section*{Problem 1}
	\emph{A chest has 3 drawers. Each drawer has 2 boxes. The boxes of one drawer contain a silver coin in each respectively. The boxes in one drawer both contain a silver coin respectively, the boxes in another drawer both contain a gold coin respectively, and in the last drawer one box contain s a silver coin and the other a gold coin. A drawer is then selected at random, and then a box is also selected at random such that it is then opened. The coin is found to be silver. What is the probability that the coin in the other box is gold?}

\pagebreak

\section*{Problem 2: WMS 2.54}
	\emph{A group of three undergraduate students and five graduate students are available to fill certain student government posts. If four students are to be randomly selected from this group, find the probability that exactly two undergraduate students will be among the four chosen.}
	
	\begin{sol}
		We have 4 positions available and 8 students in total. Therefore, the total number of position fillings is
		$$ C_4^8 = \frac{8!}{4! (8-4)!} = 70$$
		We now consider that the 8 students are separated into 3 undergarduates and 5 graduates. Among the 4 available positions, exactly 2 must be undergraduates. Therefore, the number of psoition fillings in which this holds is:
		$$ C_2^3 \times C_2^5 = \frac{3!}{2! (3-2)!} \times  \frac{5!}{2!(5-2)!} = 30$$
		We can now calculate the probability that exactly two positions are filled by undergraduates, given that there are 3 undergraduates and 5 graduates available:
		$$ \frac{C_2^3 \times C_2^5}{C_4^8} = \frac{30}{70} = \boxed{\frac37}$$
		
	\end{sol}

\pagebreak

\section*{Problem 3: WMS 2.56}
	\emph{A student prepares for an exam by studying a list of ten problems. She can solve six of them. For the exam, the instructor selects five problems at random from the ten on the list given to the students. What is the probability that the student can solve all five problems on the exam?}
	\begin{sol}
		What is the probability that the student solves 5 out of 10 randomly chosen problems correctly, given that she can solve 6 out of the 10 problems correctly?
		
		First we find the possible combinations of 5 out of 10 problems:
		
		$$	C_5^{10} = \frac{10!}{5! (10-5)!} = 252$$
		
		Now we calculate the possible ways that out of 6 problems, she can correctly solve 5 problems:
		
		$$ C_5^6 = \frac{6!}{5! (6-5)!} = 6$$
		
		Thus, the probability that she will get 5 out of 10 randomly selected problems right, given that she knows 6 problems correctly, is:
		
		$$ \frac{C_5^6}{C_5^{10}} = \frac{6}{252} = \boxed{\frac{1}{42}}$$
	\end{sol}	

\pagebreak

\section*{Problem 4: WMS 2.97}
	\emph{Consider the following portion of an electric circuit with three relays. Current will flow from point $a$ to point $b$  if there is at least one closed path when the relays are activated. The relays may malfunction and not close when activated. Suppose that the relays act independently of one another, and close properly when activated, with a probability of 0.9.
	\begin{enumerate}[$\quad\quad$a)]
		\item What is the probability that the current will flow when the relays are activated?
		\item Given that current flowed when the relays were activated, what is the probability that relay functioned?
	\end{enumerate}}

	\begin{sol}
		\begin{enumerate}[$\quad\quad$a)]
			\item Let $P(A) := $ The probability that the current flows when all the relays are activated. $P(A)$ can be given by first finding the probability that the current will not flow when all the relays are activated, $P(A^C)$, and subtract it from 1:
			\begin{align*}
				P(A) &= 1 - P(A^C)\\
				&= 1 - \big( (1-0.9) \times (1-0.9) \times (1-0.9) \big) \\
				&= 1 - (0.1)^3 \\
				&= 1 - 0.001 \\
				&= \boxed{0.999}
			\end{align*}
		
			\item Now we let $A$ be the event where the current is flowing, and $B$ be the event where relay 1 is closed properly. We know that in any case where relay 1 is closed properly, the current flows, so $B \subset A$. Moreover, $P(A \cap B) = P(B)$. Now, we can find the probability that relay 1 functioned given that the current flows when the relays are activated $P(B \mid A)$:
			$$ P(B \mid A) = \frac{P(A \cap B)}{P(A)} = \frac{0.9}{0.999} = \boxed{\frac{100}{111}}$$
		\end{enumerate}
	\end{sol}

\pagebreak

\section*{Problem 5: WMS 2.101}
	\emph{Articles coming through an inspection line are visually inspected by two successive inspectors. When a defective article comes through the inspection line, the probability that it gets by the first inspector is 0.1, then the second inspector will "miss" 5 out of 10 defective items that get past the first inspector. What is the probability that a defective article gets past both inspectors?}
	
	\begin{sol}
		Let us denote $A$ as the event where the defective article passes through the first inspector, and let $B$ denote the event where it passes through the second inspector. We know that $P(A) = 0.1$ and $P(B \mid A) = \frac{5}{10} = 0.5$, so the probability that a defective item gets past both inspectors is given by:
		$$ P(A \cap B) = P(B \mid A) \times P(A) = 0.5 \times 0.1 = \boxed{0.05} $$
	\end{sol}

\pagebreak

\section*{Problem 6: WMS 2.102}
	\emph{Diseases I and II are prevalent among people in a certain population. It is assumed that 10\% of the population will contract disease I within their lifetime, while 15\% will eventually contract disease II. Only 3\% however will contract both disease I and II.
	\begin{enumerate}[$\quad\quad$a)]
		\item Find the probability that a randomly chosen person from this population will contract at least one disease.
		\item Find the conditional probability that a randomly chosen person from this population will contract both diseases, given that he or she has contracted at least one disease before.
	\end{enumerate}}
	\begin{sol}
		\begin{enumerate}[$\quad\quad$ a)]
			\item Let us denote $A$ as the event where a person contracts disease I, and event $B$ as the event where a person contracts disease II. We know that $P(A) = 0.1$, $P(B) = 0.15$, and $P(A \cap B) = 0.03$. If we want to find the probability that a randomly chosen person from this population will contract at least one disease, we have:
			\begin{align*}
				P(A \cup B) &= P(A) + P(B) - P(A \cap B) \\
				&= 0.1 + 0.15 - 0.03 \\
				&= \boxed{0.22}
			\end{align*}
		
			\item If we want the conditional probability that a randomly chosen person from the population will contract both diseases given that she has contracted at least one disease before, we have:
			$$ P\big( (A \cap B) | (A \cup B) \big)  = \frac{P(A \cap B)}{P(A \cup B)} = \frac{0.03}{0.22} = \boxed{\frac{3}{22}}$$
			
		\end{enumerate}
	\end{sol}

\pagebreak

\section*{Problem 7: WMS 2.104}
	\emph{If $A$ and $B$ are two events, prove that $P(A \cap B) \geq 1 - P(A^C)  - P(B^C) $ \\ (Note: this is a simplified version of the \textbf{Bonferroni inequality})}
	
	\begin{proof}
		We know that $P (A \cap B) = 1 - P (A \cap B)^C = 1 - P (A^C \cup B^C)$. We also know that $P(A^C \cup B^C) \leq P(A^C) + P(B^C)$. Thus, 
		$$ P(A \cap B) \geq 1 - (P(A^C) + P(B^C) ) = 1- P(A^C) - P(B^C) $$
	\end{proof}

\pagebreak

\section*{Problem 8: WMS 2.118}
	\emph{An accident victim will die in the next 10 minutes unless he receives a blood transfusion of type A, Rh-positive blood from a single donor. The hospital requires 2 minutes to test the type of a potential donor's blood and then 2 minutes to complete the transfer of blood. Many donors with untested blood are available, and 40\% of these untested donors have type A, Rh-positive blood. What is the probability that the accident victim will be saved if only one blood-testing kit is available? Assume that the type-testing kit is reusable but can be used with only one donor at a time. }
	
	\begin{sol}
		For the accident victim to live, we must find a proper donor within at leats 8 minutes, such that the remaining 2 minutes are used to transfer the blood to the victim. As the blood testing machine takes 2 minutes to test each possible donor, a maximum of 4 attempts is possible in 8 minutes. A second attempt implies a failed first attempt, a third attempt implies failed second and first attempts, and so on. Therefore, we can deduce the probability that the victim is kept alive by letting $A$ be the event where the correct donor is found. Now, the probability $P(S)$ that the accident victim will be saved if only one blood-testing kit is available is given by: 
		\begin{align*}
			P(S) &= P(A) + P(A^C)P(A) + P(A^C)P(A^C)P(A) + P(A^C)P(A^C)P(A^C)P(A)
		\intertext{We assume that each attempt is independent, such that:}
			P(S) &= 0.4 + (0.6)(0.4) + (0.6)^2 (0.4) + (0.6)^3 (0.4) \\
			&= \boxed{\frac{544}{625}}
		\end{align*}
	\end{sol}

\pagebreak

\section*{Problem 9: WMS 2.120}
	\emph{Suppose that two defective refrigerators have been included in a shipment of six refrigerators. The buyer begins to test the six refrigerators one at a time.
	\begin{enumerate}[$\quad\quad$a)]
		\item What is the probability that the last defective refrigerator is found on the fourth test? 
		\item What is the probability that no more than four refrigerators need to be tested to locate both of the defective refigerators?
		\item When given that exactly one of the two defective refrigerators has been located in the first two test, what is the probability that the remaining defective refrigerator is found in the third or fourth test?
	\end{enumerate}
	}
	
	\begin{sol}
		\begin{enumerate}[$\quad\quad$a)]
			\item We denote $A$ as the event where a good refrigerator is found, and $B$ as the event where a defective refrigerator is found. We assume that the last defective refrigerator is found on the $4^{th}$ test, then we know that the first defective refrigerator was found on one of the first 3 tests. Hence, we have the following set of possibilities:
			
			$$ \{ BAAB, ABAB, AABB\} $$	
			We can calculate the probability that the last defective refrigerator is found on the fourth test by:
			
			$$ P \big( \{ BAAB, ABAB, AABB\}  \big) = \frac{2}{6} \cdot \frac45 \cdot\frac34 \cdot \frac13 + \frac46 \cdot\frac25 \cdot\frac34 \cdot\frac13 + \frac46\cdot \frac35 \cdot\frac24 \cdot\frac13 = \frac15$$
		\end{enumerate}
	\end{sol}
	
\end{document}  

	