\documentclass{article}

\usepackage{Style_Thesis-Report}

\title{Math 323
	- Assignment 3}
\author{Sada S\'{o}lomon, Ignacio - Id 260708051}
\date{Winter 2020}  %Place'%' at front to activate today's date


\begin{document}
\clearpage\maketitle
\thispagestyle{empty}
\vspace{2cm}

\newpage
\setcounter{page}{1}
\cfoot{\thepage}

\pagebreak

\section*{Problem 1: WMS 2.158}
\emph{A bowl contains $w$ white balls and $b$ black balls. One ball is selected at random from the bowl, and its color is noted, and it is returned to the bowl along with $n$ additional balls of the same color. Another single ball is randomly selected from the bowl (now containing $w+b+n$ balls) and it is observed that the ball is black. Show that the (conditional) probability that the first ball selected was white is} $\frac{w}{w+b+n}$
\begin{proof}
	Let us denote $W$ as the event where the first ball selected at random is white, and let $B$ denote the event where the second ball selected at random is black. Then, by \textbf{Baye's Rule}
	\begin{align*}
		P(W \mid B) &= \frac{P (B \mid W) P (W)}{P(B \mid W) P (W) + P (B \mid W^c) P(W^c)}
	\end{align*}
	We now proceed to list the value of each probability to compute $P(W \mid B)$:
	\begin{multicols}{2}
		\begin{itemize}
			\item $P(B \mid W)  = \frac{b}{w+b+n}$
			\item $P (W) = \frac{w}{w+b}$
			\item $P (B \mid W^c) = \frac{b+n}{w+b+n}$
			\item $P (W^c) = \frac{b}{w+b}$
		\end{itemize}
	\end{multicols}
	Hence,
	\begin{align*}
	P(W \mid B) &= \frac{\left( \frac{b}{w+b+n} \right) \left(\frac{w}{w+b}\right)}{\left( \frac{b}{w+b+n} \right) \left(\frac{w}{w+b}\right) + \left( \frac{b+n}{w+b+n} \right) \left(\frac{b}{w+b}\right)}\\
	&= \frac{\left( \frac{\cancel{b}w}{w+b+n} \right) \cancel{\left(\frac{1}{w+b}\right)}}{\left(  \left(\frac{\cancel{b}w}{w+b+n}\right) + \left(\frac{b^{\cancel{2}}+\cancel{b}n}{w+b+n}\right) \right) \cancel{\left( \frac{1}{w+b} \right)}} \\
	&= \frac{\left(\frac{w}{w+b+n}\right)}{\left( \frac{\cancel{w+b+n}}{\cancel{w+b+n}} \right)} \\
	&= \frac{w}{w+b+n}
	\end{align*}
\end{proof}

\pagebreak

\section*{Problem 2: WMS 2.160}
	\emph{A machine for producing a new experimental electronic component generates defectives from time to time in a random manner. The supervising engineer for a particular machine has noticed that defectives seem to be grouping (hence, appearing in a nonrandom order), thereby suggesting a malfunction in some part of the machine. One test for non-randomness is based on the number of \textbf{runs} of defectives and non-defectives (a run being an unbroken sequence of either defectives or non-defectives). The smaller the number of runs, the greater will be the amount of evidence indicating non-randomness. Of 12 components drawn from the machine, the first 10 were not defective, and the last two were defective: 
	$$ (NNNNNNNNNNDD)$$  Assume randomness.}
	\begin{enumerate}[$\quad\quad$(a)]
		\item What is the probability of observing this arrangement (resulting in two runs) given that 10 out of 12 components are non-defective?
		\item What is the probability of observing two runs? 
	\end{enumerate}
	\begin{sol}
		\begin{enumerate}[$\quad \quad$(a)]
			\item We have 12 components in total, where 10 are non-defective $(N)$ and 2 are defective $(D)$. We must first find the total number of possible arrangements for 2 defective . Thus:
			$$ C_2^{12} = \frac{12!}{2!(10!)} = \frac{11\times12}{1\times2} = 66$$
			As this specific arrangement is only once case  out of all possible cases, we have that the probability of observing this arrangement is $\frac{1}{66}$. 
			
			\item Knowing that the probability of the above arrangement is $\frac{1}{66}$, and also observing that a second possible arrangement with two runs is
			$$ (DDNNNNNNNNNN)$$
			we now simply add the probability of both to denote the probability of an arrangement with two runs, in the context of 10 non-defective and 2 defective components:
			$$ \frac{1}{66} + \frac{1}{66} = \frac{2}{66} = \frac{1}{33}$$
		\end{enumerate}
	\end{sol}
	
	\pagebreak
	
	\section*{Problem 3: WMS 2.178}
		\emph{Suppose that the probability of exposure to the flu during an epidemic is 0.6. Experience has shown that the serum is 80\% successful in preventing an inoculated person from acquiring the flu, if exposed to it. A person not inoculated faced a probability of 0.90 of acquiring the flu if exposed to it. Two persons, one inoculated and one not, perform a highly specialized task in a business. Assume that they are not at the same location, are not in contact with the same people, and cannot expose each other to the flu (they're completely independent). What is the probability that at least one of them will get the flu?}
		
	\begin{sol}
		We begin by defining $A = $ ``Person is exposed to flu" and $B=$``Person gets the flu". We now consider the case where two employees, one of whom is inoculated, and the other is not. The probability that we are looking for is the probability that at least one of the two employees catches the flu. If we define $B_1= $``Employee 1 gets the flu" and $B_2=$``Employee 2 gets the flu", as well as $A_1=$``Employee 1 is exposed to the flu" and $A_2 =$``Employee 2 is exposed to the flu", then we know that
		$$ P (B_1 \cup B_2) = 1 - P (B_1 \cup B_2)^c = 1 - P(B_1^c \cap B_2^c)$$
		If we consider employee 1 to be inoculated while employee 2 is not, then:
		$$ P(B_1^c)= P (B_1^c \cap A_1) + P (B_1^c \cap A_1^c) = 0.80 \cdot 0.60 + 1.00 \cdot 0.40 = 0.88 $$
		$$ P(B_2^c) = P(B_2^c \cap A_2) + P(B_2^c \cap A_2^c) = 0.10 \cdot 0.60 + 1.00 \cdot 0.40 = 0.46 $$
		We know that both employees and their possibilities of exposure and contracting the flu are independent of each other, thus:
		$$ P (B_1^c \cap B_2^c) = P (B_1^c) P(B_2^c) = 0.88 \cdot 0.46 = 0.4048$$
		$$ \therefore P(B_1 \cup B_2) = 1 - 0.4048 = 0.5952$$
		
	\end{sol}
\pagebreak
	\section*{Problem 4: WMS 3.4}
	\emph{Consider a system of water flowing through valves from point $A$ to point $B$. Valves 1, 2, and 3 operate independently, and each correctly opens on signal with a probability 0.8. Find the probability distribution for $Y$, the number of open paths from $A$ to $B$ after the signal is given. (Note that $Y$ can take on the values 0, 1, and 2). Also, find the \textbf{cdf} of $Y$.} 
	\begin{center}
		\begin{tikzpicture}
		\draw (0,0) -- (1,0) -- (1, -1) -- (5, -1) -- (5, 0) -- (6,0) ;
		\draw (1,0) -- (1,1) -- (5,1)  -- (5,0) ;
		\draw (0,0) node[left]{$A$} (6,0) node[right]{$B$};
		\draw 
			[fill = white] (2, -1.15) rectangle ++ (0.3, 0.3) node[below=6mm, left=-1mm]{2}
			[fill = white] (4, -1.15) rectangle ++ (0.3, 0.3) node[below=6mm, left=-1mm]{3}
			[fill = white] (3, 0.85) rectangle ++ (0.3, 0.3) node[above=3mm, left=-0.85mm]{1}
		;
		\end{tikzpicture}
	\end{center}
	\begin{sol}
		Let us define $A=$ ``Valve 1 is opened", $B=$ ``Valve 2 is opened", and $C=$ ``Valve 3 is opened".  We know that $P(Y=u) = p(u) \quad \forall  u \in \{ 0,1,2 \}$.  Thus, we must find $P(Y=0)$, $P(Y=1)$, and $P(Y=2)$, as well as $F_Y (t)$ (the \emph{cdf} of $Y$). 
		
		We begin with $P(Y=0)$, when no paths are open. This occurs if \textbf{valve 1 is not opened} \emph{and} \textbf{valve 2 or valve 3 are not opened.} Thus:
		$$ P(Y=0) = P (A^C \cap (B^c \cup C^c))$$
		Note that all valves open independently of each other, so:
		$$ P(B^c \cup C^c) = P(B \cap C)^c = 1 - (0.8)(0.8) = 0.36$$
		$$ P(A^c \cap (B^c\cup C^c)) = (0.2)(0.36) = 0.072$$
		$$ \therefore P(Y=0) = 0.072 $$
		
		Now we solve for $P(Y=2)$, where all paths are open. This occurs if \textbf{valve 1 is opened} \emph{and} \textbf{valve 2 and valve 3 are opened}. Thus:
		\begin{align*}
			P(Y=2) &= P(A \cap (B \cap C))\\
			&= P(A \cap B \cap C)\\
			&= (0.80)(0.80)(0.80) \\
			&= 0.512
		\end{align*}
		
		Finally, knowing that $u \in \{0,1,2\}$, we can find $P(Y=1)$ easily:
		\begin{align*}
			P(Y=1) &= P(\Omega) - P(Y=0) - P(Y=2)\\ 
			&= 1 - 0.072 - 0.512 \\
			&= 0.416
		\end{align*}
		
		We may now calculate the cumulative distribution function of the $Y$, considering the following:
		\begin{table}[h]
			\begin{tabular}{|c|c|c|c|}
				\hline
				$u$	&	0	&	1	&	2 \\ \hline
				$P_Y(u)$ & 0.072 & 0.416 & 0.512 \\ \hline 
			\end{tabular}
		\end{table}
	
		Thus:
		\begin{figure}[h]
			\begin{subfigure}{0.46\textwidth}
				$$F_Y (u) =  \begin{cases}
				0 & u \in (-\infty, 0) \\
				0.072 & u \in [0, 1) \\
				0. 488 & u \in [1, 2) \\
				1 & u \in [2, + \infty)
				\end{cases}$$
			\end{subfigure}
			\hfill
			\begin{subfigure}{0.46\textwidth}
				\begin{center}
					\begin{tikzpicture}
						\draw [ultra thin, grey!30] (0,0) grid (2.5,2.5);
						\draw [->] (0, 0) --  (2.5, 0) node[right]{x} ;
						\draw [->] (0, 0) --  (0, 2.5) node[above]{y} ;
						\foreach \x in {0,...,2}
							\draw (\x, -0.1) -- (\x, +0.1) node[below=2mm ]{\x};
						\foreach \y in { 0.144, 0.976, 2}
							\draw (-0.1, \y) -- (0.1, \y);
						\draw 
							(0, 0.144) node [left=2mm]{0.072}
							(0, 0.976) node[left=2mm]{0.488}
							(0, 2) node [left=2mm]{1}
						;
						\draw [line width = 2pt] (0,0.144) -- (1, 0.144);
						\draw [line width = 2pt] (1,0.976) -- (2, 0.976);
						\draw 
							(0,0) node[]{$\times$}
							(0, 0.14) node[]{$\bullet$}
							(1, 0.14) node[]{$\times$}
							(1, 0.97) node[]{$\bullet$}
							(2, 0.97) node[]{$\times$}
							(2, 2)  node[]{$\bullet$}
						;
					\end{tikzpicture}
				\end{center}
			\end{subfigure}
		\end{figure}
		Sample code for graph (using \texttt{tikz}):
		\begin{lstlisting}[language=tex]
		\begin{tikzpicture}
		% Grid and axes
		\draw [ultra thin, grey!30] (0,0) grid (2.5,2.5);
		\draw [->] (0, 0) --  (2.5, 0) node[right]{x} ;
		\draw [->] (0, 0) --  (0, 2.5) node[above]{y} ;
		
		% Scale
		\foreach \x in {0,...,2}
		\draw (\x, -0.1) -- (\x, +0.1) node[below=2mm ]{\x};
		
		% Scattering data	
		\draw [line width = 2pt] (0,0.144) -- (1, 0.144);
		\draw [line width = 2pt] (1,0.976) -- (2, 0.976);
		\draw 
		(0,0) node[]{$\times$}
		(0, 0.14) node[]{$\bullet$}
		(1, 0.14) node[]{$\times$}
		(1, 0.97) node[]{$\bullet$}
		(2, 0.97) node[]{$\times$}
		(2, 2)  node[]{$\bullet$}
		;
		\end{tikzpicture}
		\end{lstlisting}
	\end{sol}
\pagebreak
	\section*{Problem 5: WMS 3.8}
	\emph{A single cell can either die, with a probability of 0.10, or split into two, with a probability of 0.90, producing a newer generation of cells in the process. Each cell in the newer generation also dies or splits with the same probability as the initial cell. Find the probability distribution for the number of cells in the next generation, as well as the cumulative distribution function.}
	\begin{sol}
		Notice that the number of cells can never be odd as for each surviving cell, two more will result by cell division.  Thus, we let $t$ be the number of cells in each generation. Let $D=$``Cell dies" and $S=$``Cell splits". Thus:
		\begin{align*}
			P(X=0) &= P\big(D\cup (S \cap (D\cap D ))\big) \\
			&= (0.1) + (0.9) \big( (0.1)(0.1) \big) \\
			&= 0.109\\
			&\\
			P(X=4) &= P \big( S \cap (S \cap S)\big)\\
			&= (0.9)\big( (0.9)(0.9) \big) \\
			&= 0.729 \\
			&\\
			P(X=2) &= P (\Omega) - P(X=0) - P(X=4) \\
			&= 1 - 0.109 - 0.729\\
			&= 0.162
		\end{align*}
		We have the following table
		\begin{table}[h]
			\begin{tabular}{|c|c|c|c|}
				\hline
				$t$	&	0	&	2	&	4 \\ \hline
				$P_X(t)$ & 0.109 & 0.162 & 0.729 \\ \hline 
			\end{tabular}
		\end{table}
		Thus, the \emph{cdf} of this distribution is
		\begin{figure}[h]
			\begin{subfigure}{0.46\textwidth}
				$$F_X (t) =  \begin{cases}
				0 & t \in (-\infty, 0) \\
				0.109 & u \in [0, 2) \\
				0. 271 & u \in [2, 4) \\
				1 & u \in [4, + \infty)
				\end{cases}$$
			\end{subfigure}
			\hfill
			\begin{subfigure}{0.46\textwidth}
				\begin{center}
					\begin{tikzpicture}
					\draw [ultra thin, grey!30] (0,0) grid (2.5,2.5);
					\draw [->] (0, 0) --  (2.5, 0) node[right]{x} ;
					\draw [->] (0, 0) --  (0, 2.5) node[above]{y} ;
					\foreach \x in {0,...,2}
					\draw (\x, -0.1) -- (\x, +0.1);
					\draw 
					(0, 0) node [below=2mm]{0}
					(1, 0) node[below=2mm]{2}
					(2, 0) node [below=2mm]{4}
					;
					\foreach \y in { 0.218, 0.542, 2}
					\draw (-0.1, \y) -- (0.1, \y);
					\draw 
					(0, 0.218) node [left=2mm]{0.109}
					(0, 0.542) node[left=2mm]{0.271}
					(0, 2) node [left=2mm]{1}
					;
					\draw [line width = 2pt] (0,0.218) -- (1, 0.218);
					\draw [line width = 2pt] (1,0.542) -- (2, 0.542);
					\draw 
					(0,0) node[]{$\times$}
					(0, 0.212) node[]{$\bullet$}
					(1, 0.212) node[]{$\times$}
					(1, 0.54) node[]{$\bullet$}
					(2, 0.54) node[]{$\times$}
					(2, 2)  node[]{$\bullet$}
					;
					\end{tikzpicture}
				\end{center}
			\end{subfigure}
		\end{figure}
	\pagebreak
	
		Sample code for graph (using \texttt{tikz}):
		\begin{lstlisting}[language=tex]
		\begin{tikzpicture}
			\draw [ultra thin, grey!30] (0,0) grid (2.5,2.5);
			\draw [->] (0, 0) --  (2.5, 0) node[right]{x} ;
			\draw [->] (0, 0) --  (0, 2.5) node[above]{y} ;
			\foreach \x in {0,...,2}
				\draw (\x, -0.1) -- (\x, +0.1);
			\draw 
				(0, 0) node [below=2mm]{0}
				(1, 0) node[below=2mm]{2}
				(2, 0) node [below=2mm]{4}
			;
			\foreach \y in { 0.218, 0.542, 2}
				\draw (-0.1, \y) -- (0.1, \y);
			\draw 
				(0, 0.218) node [left=2mm]{0.109}
				(0, 0.542) node[left=2mm]{0.271}
				(0, 2) node [left=2mm]{1}
			;
			\draw [line width = 2pt] (0,0.218) -- (1, 0.218);
			\draw [line width = 2pt] (1,0.542) -- (2, 0.542);
			\draw 
				(0,0) node[]{$\times$}
				(0, 0.212) node[]{$\bullet$}
				(1, 0.212) node[]{$\times$}
				(1, 0.54) node[]{$\bullet$}
				(2, 0.54) node[]{$\times$}
				(2, 2)  node[]{$\bullet$}
			;
		\end{tikzpicture}
		\end{lstlisting}
	\end{sol}
	
\pagebreak

\section*{Problem 6: WMS 3.48}
	\emph{A missile protection system consists of $n$ radar sets operating independently, each with a probability of 0.90 of detecting a missile entering a zone that is covered by all the units.
	\begin{enumerate}[$\quad \quad$ (a)]
		\item If $n=5$ and a missile enters the zone, what is the probability that exactly four sets detect the missile? At least one set?
		\item  How large must $n$ be if we require that the probability of detecting a missile that enters the zone be 0.999?
	\end{enumerate}
}
	\begin{sol}
		\begin{enumerate}[$\quad \quad$ (a)]
			\item If $n=5$, we first want the probability that exactly four sets detect the missile. We notice that the distribution $X$ of number of sets that detect the missile is of binomial form:
			$$ X \sim \text{Bin}(k=5, p=0.90)$$
		 	Therefore
		 	\begin{align*}
		 		P (X=4) &= C_4^5 (0.90)^4 (1-0.90)^{5-4} \\
		 		&= \left( \frac{5!}{4! (1!)} \right) (0.9)^4 (0.1)^1 \\
		 		&= 5 \times 0.6561 \times 0.1 \\
		 		&= 0.3281
		 	\end{align*}
		 	At least 1 radar set detects a missile:
		 	\begin{align*}
		 		P (X\geq 1) &= 1 - P(X = 0) \\
		 		&= 1 - \big( C_0^5 (0.90)^0 (1-0.9)^{5-0} \big)  \\
		 		&= 1 - \bigg( \frac{5!}{0! (5!)} (1) (0.1)^{5} \bigg) \\
		 		&= 1 - (0.00001) \\
		 		&= 0.99999
		 	\end{align*}
			\item We may evaluate the probability that at least 1 out of $n$ sets detects a missile using the complement and multiplication rule of probability, derived above:
			$$ P(X \geq 1) =  1 - P (X = 0) = 1 - (0.1)^n$$
			Thus, we may now solve for $n$ considering that $P(X=n) = 0.999$:
			\begin{align*}
				0.999 &= 1- (0.1)^n\\
				\therefore 0.001 &= (0.1)^n\\
				\therefore \ln(0.001) &= n \ln(0.1) \\
				\therefore n &= \frac{\ln(0.001)}{\ln(0.1)}
			\end{align*}
			$$\therefore n = 3$$
		\end{enumerate}
	\end{sol}
	\section*{Problem 7: WMS 3.60}
	\emph{A particular concentration of a chemical found in polluted water has been found to be lethal to 20\% of the fish that are exposed to the concentration for 24 hours. Twenty fish are placed in a tank containing this concentration of chemical in the water.
	\begin{enumerate}[$\quad\quad$(a)]
		\item Find the probability that exactly 14 survive.
		\item Find the probability that at least 10 survive.
		\item  Find the probability that at most 16 survive. 
		\item Find the mean (expected value) and variance of the number that survive.
	\end{enumerate}
	}
	
	
\end{document}  

	